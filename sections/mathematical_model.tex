\documentclass[../main.tex]{subfiles}


% \newcommand*{\rttensor}[1]{\boldsymbol{\mathrm{\underline{\underline{#1}}}}}

\setlength\parindent{0cm}


\begin{document}



\chapter{Mathematical Formulation}
{\color{red}{Add full stops at the end of equations}} \\
The geometry considered is a thin film of Silicon in the domain $ -L/2 \le X \le L/2$ and $0 \le Y \le H$. Consider a certain particle, initially located at the coordinate $\X$. During deformation, this particle follows a path 


\begin{gather}
\x = \x(\X,t)
\end{gather}

\vspace{0.25cm}
Let $\u(\X, t)$ be the displacement of a material particle located at $\X$. Then
\begin{gather}
    \u(\X, t) = \x(\X, t) - \X
\end{gather}

The total deformation gradient and Green-Lagrange strain are denoted by $\F$ and $\E$, respectively. Therefore, 
\begin{align}
    \F &= \pdiff{\x}{\X} = \nablaX \u + \I \\
    \E &= \fc{1}{2}(\F^\T \cdot \F - \I)
\end{align}
where, $\I$ is the second-order isotropic tensor.

Let $\{\hat{\bm{e}}_1, \hat{\bm{e}}_2, \hat{\bm{e}}_3\}$ be the orthonormal basis in the reference configuration. Denoting the corresponding components of $\X$ by $X$, $Y$ and $Z$ and that of $\u$ by $u$, $v$ and $w$, and assuming plane strain deformation the components of F are given by:
\begin{align}
[\F] = 
\begin{bmatrix}
       1 + \pdiff{u}{X} && \pdiff{u}{Y} && 0 \\
       \pdiff{v}{X} && 1 + \pdiff{v}{Y} && 0 \\
       0 && 0 && 1
\end{bmatrix} = \tensor{F}
\end{align}
Both Inelastic and elastic deformation gradients are considered to be finite. Hence, a multiplicative decomposition of $\F$ into elastic and inelastic deformation is necessary. The body is first considered to reach an intermediate stress-free state and then it undergoes elastic deformation to reach the current configuration \citep{1969Lee}.
\begin{align}
\F = \F^{\el} \cdot  \F^{\inel} \label{eq:F_decomposition} \\
    \text{in indicial notation, }F_{ij} &= F^{\el}_{ik}F^{inel}_{kj}
\end{align}
where $\F^{\el}$ and $\F^{\inel}$ are the deformation gradients due to elastic deformation and inelastic deformation respectively.

The inelastic deformation gradient tensor, $\F^{\inel}$, has contribution from two sources - deformation due to concentration gradient, $\F^\c$, and viscoplastic deformation, $\F^\p$.
\begin{gather}
    \F^\inel = \F^\c \cdot \F^\p \label{eq:Finel_decomposition}
\end{gather}


\section{Viscoplastic Deformation}
A viscoplastic constitutive relation of the following form is considered.
\begin{gather}
\DP = \pdiff{G(\sigmaeff)}{\DCS}
\end{gather}
Where $\DP$ is the rate dependent plastic deformation tensor, $G(\sigmaeff)$ is the flow potential, $\sigmaeff$ is the von Mises stress and $\DCS$ is the deviatoric part of Cauchy stress tensor.
{\color{red}{citation}}
\begin{align}
G(\sigmaeff) &= \fc{\sigmaf \dZEROdot}{\mathrm{m}+1}\left(\fc{\sigmaeff}{\sigmaf}-1\right)^{\mathrm{m}+1}\heavi \\
\Rightarrow \mathbf{D}^{\rm{P}} &= \fc{3 \DCS \dZEROdot}{2 \sigmaeff}\left(\fc{\sigmaeff}{\sigmaf}-1\right)^{\mathrm{m}}\heavi 
\end{align}
where, H is the unit step function, $\sigmaf$ is the yield strength of Silicon, m is the stress exponent for plastic flow and $\dZEROdot$ is the strain rate for plastic flow.
\begin{align}
\mathbf{D}^{\rm{P}} &= \F^{\el} \F^{\rm{c}} \dot{\F}^\p \inv{\F^\p} \inv{\F^{\rm{c}}} \inv{\F^{\el}}  \\
\Rightarrow \dot{\F}^\p &= \inv{J} \fc{3}{2} \fc{\Mel \F^\p}{\sigmaeff} \dZEROdot \left( \fc{\sigmaeff}{\sigmaf} - 1 \right)^\m \heavi  \\
\text{where, }\Mel &= J (\F^\el)^\T \DCS (\F^\el)^\mT \\
J &= \mathrm{det}(\F) 
\end{align}
$\Mel$ is the deviatoric part of Mandel stress \citep{}. The expression for Mandel stress is $\mathbf{M}^{\el} = J (\F^{\el})^{\T} \CS \invT{F^{\el}}$.
$\F^{\rm{p}}$ is assumed to be of the following form:
\begin{gather}
    [\F^\p] = \tensor{\lambda}
\end{gather}

Since, det($\F^\p$) = 1
\begin{gather}
     \lamzz = 1/(\lamxx \lamyy - \lamxy \lamyx)
\end{gather}



\section{Deformation due to concentration gradient}
The compound formed between Lithium and Silicon is of the form $\text{Li}_{\chi}$Si. Let the stoichiometric concentration and maximum concentration of Silicon atoms per atom of Lithium be denoted by $\chi_0$ and $\chi_{\rm{max}}$. Defining a non-dimensional measure of the Li-ions concentration as $\tc  = (\chi-\chi_{0})/\xmax$. since, $\chi_{0}$ is the stoichiometric ratio it signifies the stress free state of the particle and hence, $\tc$ is a measure of the deviation of the particle from undeformed state.
\begin{align}
    \F^\c &= (\jc)^{1/3} \I \\
    \text{where } \jc &= 1 + 3 \eta \xmax \tc
\end{align}

$\eta$ is a material parameter giving rate of change in volume w.r.t. $\tc$. It may be noted that as $\tc$ approaches 1, $\mathrm{det}(\F^\c)$ approaches 4. Therefore the body undergoes a volumetric change of about 300\% due to diffusion of Li-ions, justifying the use of large deformation analysis.

\section{Momentum Conservation}
From equations \ref{eq:F_decomposition} and \ref{eq:Finel_decomposition}, $\F^{\el}$ can be expressed as, 
\begin{align}
    \F^{\el} &=  \F \cdot \inv{\F^{\p} \cdot \F^{\c}}
\end{align}
The elastic Green-Lagrange strain, 
\begin{math}
\E^{\el} = \fc{1}{2} \left[ (\F^\el)^\T \cdot  \F^\el - \I \right]
\end{math}


The strain energy per unit volume in the reference configuration, $W(\F, \tc)$, is expressed as 
\begin{math}
    W(\F, \tc) = J^{\inel} \bar{w}(\F, \tc)
\end{math}, where $\bar{w}(\F, \tc)$ is the strain energy per unit volume in the intermediate configuration and $J^{\inel} = \mathrm{det}(\F^{\inel}) = \jc$.
\begin{align}
    W(\F, \tc) &= \fc{\jc}{2}\fc{E(\tc)}{1 + \nu}\left( \fc{\nu}{1-2\nu}(\text{tr}\E^\el)^2 + \tr{\E^\el \cdot  \E^\el}\right).
\end{align}
The elastic modulus of Silicon is concentration dependent with $E(\tc) = E_{\text{si}}(1 + \eta_{\rm{E}} \chi_{\rm{max}} \tc)$. The elastic second Piola-Kirchhoff stress is denoted by $\PKS^\el$. Differentiating $W$ w.r.t $\E^\el$ gives,
\begin{gather}
    \PKS^\el = \jc[2 \muSic \E^\el + \lamSic \tr{\E^\el}\I ]
\end{gather}
Let $\PK$ and $\PKS$ denote the first and second Piola-Kirchhoff stress, respectively. Thus
\begin{align}
    \PKS &= \inv{\F^\c} \cdot  \inv{\F^\p} \cdot \PKS^\el \cdot  \invT{\F^\p} \cdot \invT{\F^\c} \\
    \PK &= \F \cdot  \PKS 
\end{align}
The Cauchy stress tensor, $\CS$ is given by 
\begin{math}\CS = \inv{J} \PK \cdot  \F^\T\
\end{math}. And, the deviatoric part of Cauchy is \begin{math}\DCS = \CS - (1/3)\tr{\CS}\I
\end{math}.
The von Mises stress is 
\begin{math}
    \sigmaeff = \sqrt{\fc{3}{2}{(\tau^2_{11} + \tau^2_{22} + \tau^2_{33}} + 2\tau^2_{12})}
\end{math}.

Conservation of momentum leads to 
\begin{gather}
\bm{\nabla}_\X \cdot \PK = 0.
\end{gather}

\section{Mass Conservation}
Assuming flux to be negligible in the $z$ direction, the conservation of mass is given by
\begin{gather}
    \pdiff{c}{t} = - \bm{\nabla}_\X \cdot \bm{j} = -\left(\pdiff{j_X}{X} + \pdiff{j_Y}{Y}\right).
\end{gather}
Where, $\bm{j}$ is the flux vector and $c$ is a dimensional measure of Li-ions concentration, defined as 
\begin{math}
    c = \xmax/\vmb \, \tc.
\end{math}
$\bm{j}$ is given by,
\begin{gather}
    \bm{j} = - \fc{1}{R_g T} \fc{D\xmax \tc}{V^b_m} \inv{\F} \invT{\F} \nablaX \mu
\end{gather}.
$\mu$ is the potential.
\begin{align}
    \mu &= \mu_0 + \mu_s \\
    \mu_0 &= R_g T \text{log}(\gamma \td{c}) \\
    \mu_s &= \fc{V_m^b}{\xmax}\left[-\fc{1}{3}\pdiff{\jc}{\tc}\tFel{im}\tFel{in}{C}_{mnkl}\tEel{kl} + \fc{1}{2}\left(\jc \pdiff{C_{ijkl}}{\tc} + \pdiff{\jc}{\tc} C_{ijkl}\right)\tEel{ij}\tEel{kl}\right] \\
    D &= D_0 \rm{exp}(\fc{\alpha S_h}{E_0}) = D_0 \rm{exp}\left(\alpha \frac{{S}_{11}+{S}_{33}}{2 E_0}\right) \\ 
    \gamma &= \fc{1}{1-\td{c}}\text{exp}(\fc{1}{R_g T}[2(A_0 - 2B_0)\td{c} - 3(A_0 - B_0)(\td{c}^2)])
\end{align}
\section{Non-Dimensionalization}
\begin{align}
    \td{j}_X, \td{j}_Y, \td{J}_0, \td{\bm{j}} &=  \fc{H \vmb}{(\xmax D_0)} (j_X, j_y, J_0, \bm{j}) \\
    \tX, \tY, \tu, \tv &= \fc{1}{H}(X, Y, u, v) \\
    \td{t} &= D_0 t/H^2 \\
     \tmusi, \tlamsi, \td{E}_{\text{si}} &= \fc{1}{E_0} (\mu_{\text{si}}, \lambda_{\text{si}}, E_{\text{si}}) 
     \text{,where }  E_0 = \fc{R_g T}{\vmb} \\
     \td{\mu}_0, \td{\mu}_1, \td{\mu}_2, \td{\mu}_3 &= \fc{1}{R_g T}(\mu_0, \mu_1, \mu_2, \mu_3) \\
     \td{D} &= \fc{D}{D_0} \\
     \dot{\td{d}}_{0} &= \fc{\dZEROdot H^2}{D_0} \\
     \td{\PKS}^{\el}, \td{\PKS}, \td{\PK}, \td{\CS}, \td{\DCS}, \td{\mathbf{M}}^{\el}_{0}, \td{\sigma}_{\rm{eff}}, \td{\sigma}_{\rm{f}} &= \fc{1}{E_0}(\PKS^{\el}, \PKS, \PK, \CS, \DCS, \mathbf{M}^{\el}_{0}, \sigmaeff, \sigmaf)
\end{align}

\section{Definition of the state of charge}
state of charge is a measure of the degree of lithiation. It can be expressed as an average concentration over the domain as follows:
\begin{align}
    \rm{soc} &= \fc{\int_{-L/2}^{L/2} \int_0^H \tc \rm{d}y \rm{d}x}{L H} \\
         &= H^2\fc{\int_{-L/2H}^{L/2H} \int_0^1 \tc(\td{x}, \td{y}) \rm{d}\td{y} \rm{d}\td{x}}{L H} \\
         &= H\fc{\int_{-L/2H}^{L/2H} \int_0^1 \tc(\td{x}, \td{y}) \rm{d}\td{y} \rm{d}\td{x}}{L} 
\end{align}

\section{Non-Dimensional Equations in Component Form}
\begin{align}
[\F] = 
\begin{bmatrix}
       1 + \pdiff{\tu}{\tX} && \pdiff{\tu}{\tY} && 0 \\
       \pdiff{\tv}{\tX} && 1 + \pdiff{\tv}{\tY} && 0 \\
       0 && 0 && 1
\end{bmatrix} = \begin{bmatrix}
    F_{11} && F_{12}  && 0 \\
    F_{21} && F_{22}  && 0 \\
    0 && 0 && F_{33}
\end{bmatrix}
\end{align}

\begin{align}
Fel_{11} &= \frac{F_{11}\,\lambda _{22}-F_{12}\,\lambda _{21}}{{J^{c}}^{1/3}\,\left(\lambda _{11}\,\lambda _{22}-\lambda _{12}\,\lambda _{21}\right)} \\ 
Fel_{21} &= \frac{F_{21}\,\lambda _{22}-F_{22}\,\lambda _{21}}{{J^{c}}^{1/3}\,\left(\lambda _{11}\,\lambda _{22}-\lambda _{12}\,\lambda _{21}\right)} \\ 
Fel_{12} &= -\frac{F_{11}\,\lambda _{12}-F_{12}\,\lambda _{11}}{{J^{c}}^{1/3}\,\left(\lambda _{11}\,\lambda _{22}-\lambda _{12}\,\lambda _{21}\right)} \\ 
Fel_{22} &= -\frac{F_{21}\,\lambda _{12}-F_{22}\,\lambda _{11}}{{J^{c}}^{1/3}\,\left(\lambda _{11}\,\lambda _{22}-\lambda _{12}\,\lambda _{21}\right)} \\ 
Fel_{33} &= \frac{F_{33}}{{J^{c}}^{1/3}\,\lambda _{33}} 
\end{align}


\begin{align}
E^{\mathrm el}_{11} &= \frac{{F_{11}^{\mathrm{el}}}^2}{2}+\frac{{F_{21}^{\mathrm{el}}}^2}{2}-\frac{1}{2} \\ 
E^{\mathrm el}_{21} &= \frac{F_{11}^{\mathrm{el}}\,F_{12}^{\mathrm{el}}}{2}+\frac{F_{21}^{\mathrm{el}}\,F_{22}^{\mathrm{el}}}{2} \\ 
E^{\mathrm el}_{12} &= \frac{F_{11}^{\mathrm{el}}\,F_{12}^{\mathrm{el}}}{2}+\frac{F_{21}^{\mathrm{el}}\,F_{22}^{\mathrm{el}}}{2} \\ 
E^{\mathrm el}_{22} &= \frac{{F_{12}^{\mathrm{el}}}^2}{2}+\frac{{F_{22}^{\mathrm{el}}}^2}{2}-\frac{1}{2} \\ 
E^{\mathrm el}_{33} &= \frac{{F_{33}^{\mathrm{el}}}^2}{2}-\frac{1}{2} 
\end{align}


\begin{align}
tSel_{11} &= J^{c}\,\left(2\,E_{11}^{\mathrm{el}}\,\tilde{\mu }_{\mathrm{si}}+\tilde{\lambda }_{\mathrm{si}}\,\left(E_{11}^{\mathrm{el}}+E_{22}^{\mathrm{el}}+E_{33}^{\mathrm{el}}\right)\right) \\ 
tSel_{21} &= 2\,E_{21}^{\mathrm{el}}\,J^{c}\,\tilde{\mu }_{\mathrm{si}} \\ 
tSel_{12} &= 2\,E_{12}^{\mathrm{el}}\,J^{c}\,\tilde{\mu }_{\mathrm{si}} \\ 
tSel_{22} &= J^{c}\,\left(2\,E_{22}^{\mathrm{el}}\,\tilde{\mu }_{\mathrm{si}}+\tilde{\lambda }_{\mathrm{si}}\,\left(E_{11}^{\mathrm{el}}+E_{22}^{\mathrm{el}}+E_{33}^{\mathrm{el}}\right)\right) \\ 
tSel_{33} &= J^{c}\,\left(2\,E_{33}^{\mathrm{el}}\,\tilde{\mu }_{\mathrm{si}}+\tilde{\lambda }_{\mathrm{si}}\,\left(E_{11}^{\mathrm{el}}+E_{22}^{\mathrm{el}}+E_{33}^{\mathrm{el}}\right)\right) 
\end{align}


\begin{align}
tS_{11} &= \frac{\tilde{S}_{11}^{\mathrm{el}}\,{\lambda _{22}}^2+\tilde{S}_{22}^{\mathrm{el}}\,{\lambda _{12}}^2-\tilde{S}_{12}^{\mathrm{el}}\,\lambda _{12}\,\lambda _{22}-\tilde{S}_{21}^{\mathrm{el}}\,\lambda _{12}\,\lambda _{22}}{{J^{c}}^{2/3}\,{\left(\lambda _{11}\,\lambda _{22}-\lambda _{12}\,\lambda _{21}\right)}^2} \\ 
tS_{21} &= \frac{\tilde{S}_{12}^{\mathrm{el}}\,\lambda _{12}\,\lambda _{21}-\tilde{S}_{22}^{\mathrm{el}}\,\lambda _{11}\,\lambda _{12}-\tilde{S}_{11}^{\mathrm{el}}\,\lambda _{21}\,\lambda _{22}+\tilde{S}_{21}^{\mathrm{el}}\,\lambda _{11}\,\lambda _{22}}{{J^{c}}^{2/3}\,{\left(\lambda _{11}\,\lambda _{22}-\lambda _{12}\,\lambda _{21}\right)}^2} \\ 
tS_{12} &= \frac{\tilde{S}_{12}^{\mathrm{el}}\,\lambda _{11}\,\lambda _{22}-\tilde{S}_{22}^{\mathrm{el}}\,\lambda _{11}\,\lambda _{12}-\tilde{S}_{11}^{\mathrm{el}}\,\lambda _{21}\,\lambda _{22}+\tilde{S}_{21}^{\mathrm{el}}\,\lambda _{12}\,\lambda _{21}}{{J^{c}}^{2/3}\,{\left(\lambda _{11}\,\lambda _{22}-\lambda _{12}\,\lambda _{21}\right)}^2} \\ 
tS_{22} &= \frac{\tilde{S}_{11}^{\mathrm{el}}\,{\lambda _{21}}^2+\tilde{S}_{22}^{\mathrm{el}}\,{\lambda _{11}}^2-\tilde{S}_{12}^{\mathrm{el}}\,\lambda _{11}\,\lambda _{21}-\tilde{S}_{21}^{\mathrm{el}}\,\lambda _{11}\,\lambda _{21}}{{J^{c}}^{2/3}\,{\left(\lambda _{11}\,\lambda _{22}-\lambda _{12}\,\lambda _{21}\right)}^2} \\ 
tS_{33} &= \frac{\tilde{S}_{33}^{\mathrm{el}}}{{J^{c}}^{2/3}\,{\lambda _{33}}^2} 
\end{align}


\begin{align}
\tilde{P}_{11} &= F_{11}\,\tilde{S}_{11}+F_{12}\,\tilde{S}_{21} \\ 
\tilde{P}_{21} &= F_{21}\,\tilde{S}_{11}+F_{22}\,\tilde{S}_{21} \\ 
\tilde{P}_{12} &= F_{11}\,\tilde{S}_{12}+F_{12}\,\tilde{S}_{22} \\ 
\tilde{P}_{22} &= F_{21}\,\tilde{S}_{12}+F_{22}\,\tilde{S}_{22} \\ 
\tilde{P}_{33} &= \tilde{S}_{33} 
\end{align}


\begin{align}
\tilde{\sigma}_{11} &= \frac{F_{11}\,\tilde{P}_{11}+F_{12}\,\tilde{P}_{12}}{J} \\ 
\tilde{\sigma}_{21} &= \frac{F_{11}\,\tilde{P}_{21}+F_{12}\,\tilde{P}_{22}}{J} \\ 
\tilde{\sigma}_{12} &= \frac{F_{21}\,\tilde{P}_{11}+F_{22}\,\tilde{P}_{12}}{J} \\ 
\tilde{\sigma}_{22} &= \frac{F_{21}\,\tilde{P}_{21}+F_{22}\,\tilde{P}_{22}}{J} \\ 
\tilde{\sigma}_{33} &= \frac{\tilde{P}_{33}}{J} 
\end{align}


\begin{align}
ttau_{11} &= \frac{2\,\tilde{\sigma }_{11}}{3}-\frac{\tilde{\sigma }_{22}}{3}-\frac{\tilde{\sigma }_{33}}{3} \\ 
ttau_{21} &= \tilde{\sigma }_{21} \\ 
ttau_{12} &= \tilde{\sigma }_{12} \\ 
ttau_{22} &= \frac{2\,\tilde{\sigma }_{22}}{3}-\frac{\tilde{\sigma }_{11}}{3}-\frac{\tilde{\sigma }_{33}}{3} \\ 
ttau_{33} &= \frac{2\,\tilde{\sigma }_{33}}{3}-\frac{\tilde{\sigma }_{22}}{3}-\frac{\tilde{\sigma }_{11}}{3} 
\end{align}



\begin{align}
    \tsigmaeff &= \sqrt{\fc{3}{2}(\ttau{11}^2 + \ttau{22}^2 + \ttau{33}^2 + 2\ttau{12}^2) }
\end{align}


\begin{align}
tMel_{11} &= -\frac{J\,\left(F_{11}^{\mathrm{el}}\,F_{12}^{\mathrm{el}}\,\tilde{\tau }_{12}-F_{11}^{\mathrm{el}}\,F_{22}^{\mathrm{el}}\,\tilde{\tau }_{11}+F_{12}^{\mathrm{el}}\,F_{21}^{\mathrm{el}}\,\tilde{\tau }_{22}-F_{21}^{\mathrm{el}}\,F_{22}^{\mathrm{el}}\,\tilde{\tau }_{21}\right)}{F_{11}^{\mathrm{el}}\,F_{22}^{\mathrm{el}}-F_{12}^{\mathrm{el}}\,F_{21}^{\mathrm{el}}} \\ 
tMel_{21} &= -\frac{J\,\left({F_{12}^{\mathrm{el}}}^2\,\tilde{\tau }_{12}-{F_{22}^{\mathrm{el}}}^2\,\tilde{\tau }_{21}-F_{12}^{\mathrm{el}}\,F_{22}^{\mathrm{el}}\,\tilde{\tau }_{11}+F_{12}^{\mathrm{el}}\,F_{22}^{\mathrm{el}}\,\tilde{\tau }_{22}\right)}{F_{11}^{\mathrm{el}}\,F_{22}^{\mathrm{el}}-F_{12}^{\mathrm{el}}\,F_{21}^{\mathrm{el}}} \\ 
tMel_{12} &= \frac{J\,\left({F_{11}^{\mathrm{el}}}^2\,\tilde{\tau }_{12}-{F_{21}^{\mathrm{el}}}^2\,\tilde{\tau }_{21}-F_{11}^{\mathrm{el}}\,F_{21}^{\mathrm{el}}\,\tilde{\tau }_{11}+F_{11}^{\mathrm{el}}\,F_{21}^{\mathrm{el}}\,\tilde{\tau }_{22}\right)}{F_{11}^{\mathrm{el}}\,F_{22}^{\mathrm{el}}-F_{12}^{\mathrm{el}}\,F_{21}^{\mathrm{el}}} \\ 
tMel_{22} &= \frac{J\,\left(F_{11}^{\mathrm{el}}\,F_{12}^{\mathrm{el}}\,\tilde{\tau }_{12}-F_{12}^{\mathrm{el}}\,F_{21}^{\mathrm{el}}\,\tilde{\tau }_{11}+F_{11}^{\mathrm{el}}\,F_{22}^{\mathrm{el}}\,\tilde{\tau }_{22}-F_{21}^{\mathrm{el}}\,F_{22}^{\mathrm{el}}\,\tilde{\tau }_{21}\right)}{F_{11}^{\mathrm{el}}\,F_{22}^{\mathrm{el}}-F_{12}^{\mathrm{el}}\,F_{21}^{\mathrm{el}}} \\ 
tMel_{33} &= J\,\tilde{\tau }_{33} 
\end{align}


{\color{red}{These equations need editing if Fpdot\_mat.tex is changed}}
\begin{align}
Fpdot_{11} &= \dot{d}_{0}\,{\left(\frac{\sigma _{\mathrm{eff}}}{\sigma _{f}}-1\right)}^m\,\left(\frac{3\,M_{11}^{\mathrm{el}}\,\lambda _{11}}{2\,J\,\sigma _{\mathrm{eff}}}+\frac{3\,M_{12}^{\mathrm{el}}\,\lambda _{21}}{2\,J\,\sigma _{\mathrm{eff}}}\right) \\ 
Fpdot_{21} &= \dot{d}_{0}\,{\left(\frac{\sigma _{\mathrm{eff}}}{\sigma _{f}}-1\right)}^m\,\left(\frac{3\,M_{21}^{\mathrm{el}}\,\lambda _{11}}{2\,J\,\sigma _{\mathrm{eff}}}+\frac{3\,M_{22}^{\mathrm{el}}\,\lambda _{21}}{2\,J\,\sigma _{\mathrm{eff}}}\right) \\ 
Fpdot_{12} &= \dot{d}_{0}\,{\left(\frac{\sigma _{\mathrm{eff}}}{\sigma _{f}}-1\right)}^m\,\left(\frac{3\,M_{11}^{\mathrm{el}}\,\lambda _{12}}{2\,J\,\sigma _{\mathrm{eff}}}+\frac{3\,M_{12}^{\mathrm{el}}\,\lambda _{22}}{2\,J\,\sigma _{\mathrm{eff}}}\right) \\ 
Fpdot_{22} &= \dot{d}_{0}\,{\left(\frac{\sigma _{\mathrm{eff}}}{\sigma _{f}}-1\right)}^m\,\left(\frac{3\,M_{21}^{\mathrm{el}}\,\lambda _{12}}{2\,J\,\sigma _{\mathrm{eff}}}+\frac{3\,M_{22}^{\mathrm{el}}\,\lambda _{22}}{2\,J\,\sigma _{\mathrm{eff}}}\right) \\ 
Fpdot_{33} &= \frac{3\,M_{33}^{\mathrm{el}}\,\dot{d}_{0}\,\lambda _{33}\,{\left(\frac{\sigma _{\mathrm{eff}}}{\sigma _{f}}-1\right)}^m}{2\,J\,\sigma _{\mathrm{eff}}} 
\end{align}



\begin{align}
    \bm{\td{j}} &= \frac{\bm{j} H}{\cmax D_0} =-\fc{1}{R_g T} \td{D} \tc \inv{\F} \invT{\F} \nablaX \mu \\
    D &= D_0 \rm{exp}(\fc{\alpha S_h}{E_0}) = D_0 \rm{exp}(\alpha \td{S}_h)= D_0 \rm{exp}\left(\alpha \frac{\td{S}_{11}+\td{S}_{33}}{2}\right) \\ 
    \td{\mu_0} &= \text{log}(\gamma \td{c})\\
    \gamma &= \fc{1}{1-\td{c}}\text{exp}(\fc{1}{R_g T}[2(A_0 - 2B_0)\td{c} - 3(A_0 - B_0)(\td{c}^2)])\\
    \td{\mu}_s &= \frac{1}{R_g T} \mu_s \\
        &= \fc{V_m^b}{R_g T \xmax}\left[-\fc{1}{3}\pdiff{\jc}{\tc}\tFel{im}\tFel{in}{C}_{mnkl}\tEel{kl} + \fc{1}{2}\left(\jc \pdiff{C_{ijkl}}{\tc} + \pdiff{\jc}{\tc} C_{ijkl}\right)\tEel{ij}\tEel{kl}\right] \\ 
        &= \fc{1}{\xmax}\left[-\fc{1}{3}\pdiff{\jc}{\tc}\tFel{im}\tFel{in}{\td{C}}_{mnkl}\tEel{kl} + \fc{1}{2}\left(\jc \pdiff{\td{C}_{ijkl}}{\tc} + \pdiff{\jc}{\tc} \td{C}_{ijkl}\right)\tEel{ij}\tEel{kl}\right] \\
    % \td{C}_{ijkl} &= \jc\left[\td{\lambda}_{si}(\tc)\delta_{ij} \delta_{kl} + \td{\mu}_{si}(\tc)(\delta_{ik} \delta_{jl} + \delta_{il} \delta_{jk})\right] \\
    \td{C}_{ijkl} \tEel{kl} &= \tPel{ij}  = \tSel{ij}/\jc \Longrightarrow \td{C}_{ijkl} = \td{\lambda}_{si}(\tc)\delta_{ij} \delta_{kl} +  \td{\mu}_{si}(\tc)(\delta_{ik} \delta_{jl} + \delta_{il} \delta_{jk})  \\
    \td{\mu}_s &= \fc{1}{\xmax}\left[-\fc{1}{3}\pdiff{\jc}{\tc}\tFel{im}\tFel{in}\tPel{mn} + \fc{1}{2}\jc \pdiff{\td{C}_{ijkl}}{\tc}\tEel{ij}\tEel{kl} + \fc{1}{2} \pdiff{\jc}{\tc} \td{C}_{ijkl}\tEel{ij}\tEel{kl}\right] \\
    &= \fc{1}{\xmax}\left[-\fc{1}{3}\pdiff{\jc}{\tc}\tFel{im}\tFel{in}\tPel{mn} + \fc{1}{2}\jc \pdiff{\td{C}_{ijkl}}{\tc}\tEel{ij}\tEel{kl} + \fc{1}{2} \pdiff{\jc}{\tc} \tEel{ij}\tPel{kl}\right] \\
    &= \fc{1}{\xmax}\left[\pdiff{\jc}{\tc} \tPel{mn} \left( -\fc{1}{3}\tFel{im}\tFel{in}+ \fc{1}{2} \tEel{mn}\right)+ \fc{1}{2}\jc \pdiff{\td{C}_{ijkl}}{\tc}\tEel{ij}\tEel{kl} \right] \\
    &= \fc{1}{\xmax}\left[\pdiff{\jc}{\tc} \tPel{mn} \left( -\fc{1}{3}(2\tEel{mn}+\delta_{mn})+ \fc{1}{2} \tEel{mn}\right)+ \fc{1}{2}\jc \pdiff{\td{C}_{ijkl}}{\tc}\tEel{ij}\tEel{kl} \right] \\
    &= \fc{1}{\xmax}\left[-\fc{1}{6}\pdiff{\jc}{\tc} \tPel{mn} \tEel{mn} - \fc{1}{3}\pdiff{\jc}{\tc} \tPel{mn}\delta_{mn} +  \fc{1}{2}\jc \pdiff{\td{C}_{ijkl}}{\tc}\tEel{ij}\tEel{kl} \right] \\
    &= \fc{1}{\xmax}\left[-\fc{1}{6}\pdiff{\jc}{\tc} \tPel{mn} \tEel{mn} - \fc{1}{3 }\pdiff{\jc}{\tc} \tPel{mm} +  \fc{1}{2}\jc \pdiff{\td{C}_{ijkl}}{\tc}\tEel{ij}\tEel{kl} \right] \\
    &= \frac{1}{\xmax}(\td{\mu}_1 + \td{\mu}_2 + \td{\mu}_3)
\end{align}



\begin{align}
    \td{\mu}_0 &= \text{log}(\gamma \td{c}) \\
    \td{\mu}_1 &= -\fc{1}{6}\pdiff{\jc}{\tc} [\tPel{11}\tEel{11} + \tPel{22}\tEel{22} + 2\tPel{12}\tEel{12} + \tPel{33}\tEel{33}] \\
    \td{\mu}_2 &= -\fc{1}{3}\pdiff{ \jc}{\tc}[\tPel{11} + \tPel{22} + \tPel{33}] \\
    \td{\mu}_3 &= \fc{1}{2}\jc \pdiff{ \td{C}_{ijkl}}{\tc}\tEel{kl}\tEel{ij} \\
    \pdiff{\td{C}_{ijkl}}{\tc} &= \pdiff{\td{\lambda}_{si}(\tc)}{\tc}\delta_{ij} \delta_{kl} +  \pdiff{\td{\mu}_{si}(\tc)}{\tc}(\delta_{ik} \delta_{jl} + \delta_{il} \delta_{jk})  \\
    \td{\mu}_3  &=  \fc{1}{2}\jc \pdiff{\td{\lambda}_{si}(\tc)}{\tc}\delta_{ij} \delta_{kl} \tEel{kl}\tEel{ij} + \fc{1}{2}\jc\pdiff{\td{\mu}_{si}(\tc)}{\tc}(\delta_{ik} \delta_{jl} \tEel{kl}\tEel{ij} + \delta_{il} \delta_{jk} \tEel{kl}\tEel{ij}) \\
    &=  \fc{1}{2}\jc \pdiff{\td{\lambda}_{si}(\tc)}{\tc} \tEel{kk}\tEel{ii} + \fc{1}{2}\jc\pdiff{\td{\mu}_{si}(\tc)}{\tc}(\tEel{ij}\tEel{ij} +  \tEel{ji}\tEel{ij}) \\
    &=  \fc{1}{2}\jc \pdiff{\td{\lambda}_{si}(\tc)}{\tc} (\mathrm{tr}(\mathbf{E^{\rm{el}}}))^2 + \jc\pdiff{\td{\mu}_{si}(\tc)}{\tc}\tEel{ij}\tEel{ij} \\
    &= \fc{1}{2}\jc[\tlamsi'(c)(\tEel{11} + \tEel{22} + \tEel{33})^2 + 2\tmusi'(\td{c})((\tEel{11})^2 + (\tEel{22})^2 + (\tEel{33})^2 + 2 (\tEel{12})^2)]\\
    \bm{j} &= - \fc{D\xmax \tc}{V^b_m} \td{\F}^{-1} (\td{\F}^{-1})^\T \nabla_{\X} \td \mu \\
    \bm{\td j} &= \bm{j}H V_m^b/(\chi_{\text{max}}D_0)\\
                &= -\fc{D}{D_0}H \tc \td{\F}^{-1}(\td{\F}^{-1})^\T \nabla_{\X} \td \mu\\
    \td{j}_x &= \fc{-\bar{D} H \tc}{J^2} \left(\pdiff{ \td \mu}{X}(\tF{12}^2 + \tF{22}^2) - \pdiff{ \td \mu}{ Y}(\tF{11}\tF{12} + \tF{21}\tF{22})\right)\\
    &= \fc{-\bar{D}  \tc}{J^2} \left(\pdiff{ \td \mu}{\tX}(\tF{12}^2 + \tF{22}^2) - \pdiff{ \td \mu}{\tY}(\tF{11}\tF{12} + \tF{21}\tF{22})\right)\\
    \td{j}_y &= \fc{-\bar{D} H \tc}{J^2} \left(\pdiff{ \td \mu}{Y}(\tF{11}^2 + \tF{21}^2) - \pdiff{ \td \mu}{X}(\tF{11}\tF{12} + \tF{21}\tF{22})\right)\\
    &= \fc{-\bar{D}  \tc}{J^2} \left( \pdiff{ \td \mu}{\tY}(\tF{11}^2 + \tF{21}^2) -\pdiff{ \td \mu}{\tX}(\tF{11}\tF{12} + \tF{21}\tF{22})\right)
\end{align}

Momentum Conservation equations in non-dimensional form:
\begin{align}
    \pdiff{\tP{11}}{\tX} + \pdiff{\tP{12}}{\tY} &= 0 \\
    \text{and, }   \pdiff{\tP{21}}{\tX} +  \pdiff{\tP{22}}{\tY} &= 0.
\end{align}


Mass conservation Equation in non-dimensional form:

\begin{align}
    \pdiff{c}{t} &= - \nablaX \cdot \bm{j} = -(\pdiff{j_x}{X} + \pdiff{j_y}{ Y}) \\
    \cmax \fc{D_0}{H^2} \pdiff{{\td{c}}}{\tt}  &= -(\fc{1}{H}\pdiff{ \td{j}_x}{\tX} + \fc{1}{H}\pdiff{ \td{j}_y}{\tY})\cmax \fc{D_0}{H} \\
    \Longrightarrow  \pdiff{{\tc}}{\tt}  &= - (\pdiff{ \td{j}_x}{\tX} +  \pdiff{ \td{j}_y}{\tY}) 
\end{align}

\section{Boundary and Initial Conditions}
\begin{align}
    \tc(\tX, \tY, 0) = 0 \\
    \tu(\tX, \tY, 0) = 0 \\
    \tv(\tX, \tY, 0) = 0 \\
    \tu(\tX, 0, \tt) = \tv(\tX, 0, \tt) = 0 \\
    \tu(-1/2, \tY, \tt) =  \tu(1/2, \tY, \tt) = 0 \\
    \td{j}_x(\tX, 1, \tt) = \td{j}_0 (1 - \tc(\tX, 1, \tt))
\end{align} 

% \pagebreak

% \section{Table of parameters}
% \begin{center}
% \begin{tabular}{cc}
% \hline
% Parameter & value\\
% \hline
% \end{tabular}
% \end{center}

\end{document}
